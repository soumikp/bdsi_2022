% Options for packages loaded elsewhere
\PassOptionsToPackage{unicode}{hyperref}
\PassOptionsToPackage{hyphens}{url}
%
\documentclass[
]{article}
\usepackage{amsmath,amssymb}
\usepackage{lmodern}
\usepackage{iftex}
\ifPDFTeX
  \usepackage[T1]{fontenc}
  \usepackage[utf8]{inputenc}
  \usepackage{textcomp} % provide euro and other symbols
\else % if luatex or xetex
  \usepackage{unicode-math}
  \defaultfontfeatures{Scale=MatchLowercase}
  \defaultfontfeatures[\rmfamily]{Ligatures=TeX,Scale=1}
\fi
% Use upquote if available, for straight quotes in verbatim environments
\IfFileExists{upquote.sty}{\usepackage{upquote}}{}
\IfFileExists{microtype.sty}{% use microtype if available
  \usepackage[]{microtype}
  \UseMicrotypeSet[protrusion]{basicmath} % disable protrusion for tt fonts
}{}
\makeatletter
\@ifundefined{KOMAClassName}{% if non-KOMA class
  \IfFileExists{parskip.sty}{%
    \usepackage{parskip}
  }{% else
    \setlength{\parindent}{0pt}
    \setlength{\parskip}{6pt plus 2pt minus 1pt}}
}{% if KOMA class
  \KOMAoptions{parskip=half}}
\makeatother
\usepackage{xcolor}
\IfFileExists{xurl.sty}{\usepackage{xurl}}{} % add URL line breaks if available
\IfFileExists{bookmark.sty}{\usepackage{bookmark}}{\usepackage{hyperref}}
\hypersetup{
  pdftitle={Probability review: handout},
  hidelinks,
  pdfcreator={LaTeX via pandoc}}
\urlstyle{same} % disable monospaced font for URLs
\usepackage[margin=1in]{geometry}
\usepackage{color}
\usepackage{fancyvrb}
\newcommand{\VerbBar}{|}
\newcommand{\VERB}{\Verb[commandchars=\\\{\}]}
\DefineVerbatimEnvironment{Highlighting}{Verbatim}{commandchars=\\\{\}}
% Add ',fontsize=\small' for more characters per line
\usepackage{framed}
\definecolor{shadecolor}{RGB}{248,248,248}
\newenvironment{Shaded}{\begin{snugshade}}{\end{snugshade}}
\newcommand{\AlertTok}[1]{\textcolor[rgb]{0.94,0.16,0.16}{#1}}
\newcommand{\AnnotationTok}[1]{\textcolor[rgb]{0.56,0.35,0.01}{\textbf{\textit{#1}}}}
\newcommand{\AttributeTok}[1]{\textcolor[rgb]{0.77,0.63,0.00}{#1}}
\newcommand{\BaseNTok}[1]{\textcolor[rgb]{0.00,0.00,0.81}{#1}}
\newcommand{\BuiltInTok}[1]{#1}
\newcommand{\CharTok}[1]{\textcolor[rgb]{0.31,0.60,0.02}{#1}}
\newcommand{\CommentTok}[1]{\textcolor[rgb]{0.56,0.35,0.01}{\textit{#1}}}
\newcommand{\CommentVarTok}[1]{\textcolor[rgb]{0.56,0.35,0.01}{\textbf{\textit{#1}}}}
\newcommand{\ConstantTok}[1]{\textcolor[rgb]{0.00,0.00,0.00}{#1}}
\newcommand{\ControlFlowTok}[1]{\textcolor[rgb]{0.13,0.29,0.53}{\textbf{#1}}}
\newcommand{\DataTypeTok}[1]{\textcolor[rgb]{0.13,0.29,0.53}{#1}}
\newcommand{\DecValTok}[1]{\textcolor[rgb]{0.00,0.00,0.81}{#1}}
\newcommand{\DocumentationTok}[1]{\textcolor[rgb]{0.56,0.35,0.01}{\textbf{\textit{#1}}}}
\newcommand{\ErrorTok}[1]{\textcolor[rgb]{0.64,0.00,0.00}{\textbf{#1}}}
\newcommand{\ExtensionTok}[1]{#1}
\newcommand{\FloatTok}[1]{\textcolor[rgb]{0.00,0.00,0.81}{#1}}
\newcommand{\FunctionTok}[1]{\textcolor[rgb]{0.00,0.00,0.00}{#1}}
\newcommand{\ImportTok}[1]{#1}
\newcommand{\InformationTok}[1]{\textcolor[rgb]{0.56,0.35,0.01}{\textbf{\textit{#1}}}}
\newcommand{\KeywordTok}[1]{\textcolor[rgb]{0.13,0.29,0.53}{\textbf{#1}}}
\newcommand{\NormalTok}[1]{#1}
\newcommand{\OperatorTok}[1]{\textcolor[rgb]{0.81,0.36,0.00}{\textbf{#1}}}
\newcommand{\OtherTok}[1]{\textcolor[rgb]{0.56,0.35,0.01}{#1}}
\newcommand{\PreprocessorTok}[1]{\textcolor[rgb]{0.56,0.35,0.01}{\textit{#1}}}
\newcommand{\RegionMarkerTok}[1]{#1}
\newcommand{\SpecialCharTok}[1]{\textcolor[rgb]{0.00,0.00,0.00}{#1}}
\newcommand{\SpecialStringTok}[1]{\textcolor[rgb]{0.31,0.60,0.02}{#1}}
\newcommand{\StringTok}[1]{\textcolor[rgb]{0.31,0.60,0.02}{#1}}
\newcommand{\VariableTok}[1]{\textcolor[rgb]{0.00,0.00,0.00}{#1}}
\newcommand{\VerbatimStringTok}[1]{\textcolor[rgb]{0.31,0.60,0.02}{#1}}
\newcommand{\WarningTok}[1]{\textcolor[rgb]{0.56,0.35,0.01}{\textbf{\textit{#1}}}}
\usepackage{graphicx}
\makeatletter
\def\maxwidth{\ifdim\Gin@nat@width>\linewidth\linewidth\else\Gin@nat@width\fi}
\def\maxheight{\ifdim\Gin@nat@height>\textheight\textheight\else\Gin@nat@height\fi}
\makeatother
% Scale images if necessary, so that they will not overflow the page
% margins by default, and it is still possible to overwrite the defaults
% using explicit options in \includegraphics[width, height, ...]{}
\setkeys{Gin}{width=\maxwidth,height=\maxheight,keepaspectratio}
% Set default figure placement to htbp
\makeatletter
\def\fps@figure{htbp}
\makeatother
\setlength{\emergencystretch}{3em} % prevent overfull lines
\providecommand{\tightlist}{%
  \setlength{\itemsep}{0pt}\setlength{\parskip}{0pt}}
\setcounter{secnumdepth}{-\maxdimen} % remove section numbering
\ifLuaTeX
  \usepackage{selnolig}  % disable illegal ligatures
\fi

\title{Probability review: handout}
\author{}
\date{\vspace{-2.5em}}

\begin{document}
\maketitle

\hypertarget{general-advice}{%
\subsection{General Advice}\label{general-advice}}

Although the R scripts written in these labs are relatively short, it is
still important to follow some best practices that help make the code
easier to read and modify.

\begin{itemize}
\item
  Organize the code in a clear, logical manner. Be sure that the script
  can be run from beginning to end, line-by-line, without errors.
\item
  Annotate the code with comments to make it easier to identify the
  purpose of specific sections.
\item
  Start out by defining the parameters (i.e., variables) that will be
  used in the script. Use variables as much as possible so that the code
  is easy to re-use in similar settings.
\end{itemize}

\hypertarget{introduction}{%
\subsection{Introduction}\label{introduction}}

Simulation allows for an intuitive way to understand the definition of
probability as a proportion of times that an outcome of interest occurs
if the random phenomenon could be repeated infinitely. With a
programming language like R, it is possible to feasibly conduct
simulations with enough replicates such that the proportion of
occurrences with a particular outcome closely approximates the
probability \(p\).

\hypertarget{notes}{%
\subsection{Notes}\label{notes}}

In this section (click to expand), we introduce the basic elements of R
programming required to conduct such simulations: the \texttt{sample()}
command and \texttt{for} loop control structure.

Click for notes on \texttt{sample()} and \texttt{for} loops

\hypertarget{random-sampling-with-sample}{%
\subsubsection{\texorpdfstring{Random Sampling with
\texttt{sample()}}{Random Sampling with sample()}}\label{random-sampling-with-sample}}

In a probability setting, it is not always the case that each outcome of
an experiment is equally likely. The \texttt{sample(\ )} function has
the generic structure

\begin{Shaded}
\begin{Highlighting}[]
\FunctionTok{sample}\NormalTok{(x, }\AttributeTok{size =}\NormalTok{ , }\AttributeTok{replace =} \ConstantTok{FALSE}\NormalTok{, }\AttributeTok{prob =} \ConstantTok{NULL}\NormalTok{)}
\end{Highlighting}
\end{Shaded}

where the \texttt{prob} argument allows for the probability of sampling
each element in \texttt{x} to be specified as a vector. When
\texttt{prob} is omitted, the function will sample each element with
equal probability.

The following code simulates the outcome of tossing a biased coin ten
times, where the probability of a heads is 0.6; a heads is represented
by \texttt{1} and a tails is represented by \texttt{0} The first
argument is the vector \texttt{(0,\ 1)}, and the \texttt{prob} argument
is the vector \texttt{(0.4,\ 0.6)}; the order indicates that the first
element (\texttt{0}) is to be sampled with probability 0.4 and the
second element (\texttt{1}) is to be sampled with probability 0.6.

\begin{Shaded}
\begin{Highlighting}[]
\CommentTok{\#set the seed for a pseudo{-}random sample}
\FunctionTok{set.seed}\NormalTok{(}\DecValTok{2018}\NormalTok{)}
\NormalTok{outcomes }\OtherTok{=} \FunctionTok{sample}\NormalTok{(}\FunctionTok{c}\NormalTok{(}\DecValTok{0}\NormalTok{, }\DecValTok{1}\NormalTok{), }\AttributeTok{size =} \DecValTok{10}\NormalTok{, }\AttributeTok{prob =} \FunctionTok{c}\NormalTok{(}\FloatTok{0.4}\NormalTok{, }\FloatTok{0.6}\NormalTok{), }\AttributeTok{replace =} \ConstantTok{TRUE}\NormalTok{)}
\NormalTok{outcomes}
\end{Highlighting}
\end{Shaded}

\begin{verbatim}
##  [1] 1 1 1 1 1 1 0 1 0 1
\end{verbatim}

\hypertarget{using-sum}{%
\subsubsection{\texorpdfstring{Using
\texttt{sum\ ()}}{Using sum ()}}\label{using-sum}}

The \texttt{sum\ ()} function is used in the lab to return the sum of
the outcomes vector, where outcomes are recorded as either \texttt{0} or
\texttt{1}. For simplicity, the function was used in its most basic
form:

\begin{Shaded}
\begin{Highlighting}[]
\FunctionTok{sum}\NormalTok{(outcomes)       }\CommentTok{\#number of 1\textquotesingle{}s (heads)}
\end{Highlighting}
\end{Shaded}

\begin{verbatim}
## [1] 8
\end{verbatim}

\begin{Shaded}
\begin{Highlighting}[]
\DecValTok{10} \SpecialCharTok{{-}} \FunctionTok{sum}\NormalTok{(outcomes)  }\CommentTok{\#number of 0\textquotesingle{}s (tails)}
\end{Highlighting}
\end{Shaded}

\begin{verbatim}
## [1] 2
\end{verbatim}

It is often convenient to combine the \texttt{sum()} function with
logical operators.

\begin{Shaded}
\begin{Highlighting}[]
\FunctionTok{sum}\NormalTok{(outcomes }\SpecialCharTok{==} \DecValTok{1}\NormalTok{)  }\CommentTok{\#number of 1\textquotesingle{}s (heads)}
\end{Highlighting}
\end{Shaded}

\begin{verbatim}
## [1] 8
\end{verbatim}

\begin{Shaded}
\begin{Highlighting}[]
\FunctionTok{sum}\NormalTok{(outcomes }\SpecialCharTok{==} \DecValTok{0}\NormalTok{)  }\CommentTok{\#number of 0\textquotesingle{}s (tails)}
\end{Highlighting}
\end{Shaded}

\begin{verbatim}
## [1] 2
\end{verbatim}

This provides more flexibility and can make the code easier to parse
quickly, such as for cases where there are more than two outcomes, or
when more than two outcomes are of interest. For example, the following
application of \texttt{sum()} identifies the number of rolls (out of
twenty) of a fair six-sided that are either 1 or greater than 4.

\begin{Shaded}
\begin{Highlighting}[]
\CommentTok{\#set the seed for a pseudo{-}random sample}
\FunctionTok{set.seed}\NormalTok{(}\DecValTok{2018}\NormalTok{)}
\NormalTok{dice.rolls }\OtherTok{=} \FunctionTok{sample}\NormalTok{(}\DecValTok{1}\SpecialCharTok{:}\DecValTok{6}\NormalTok{, }\AttributeTok{size =} \DecValTok{20}\NormalTok{, }\AttributeTok{replace =} \ConstantTok{TRUE}\NormalTok{)}
\NormalTok{dice.rolls}
\end{Highlighting}
\end{Shaded}

\begin{verbatim}
##  [1] 3 4 5 2 5 1 3 4 2 4 3 3 6 1 1 6 5 3 1 3
\end{verbatim}

\begin{Shaded}
\begin{Highlighting}[]
\FunctionTok{sum}\NormalTok{(dice.rolls }\SpecialCharTok{==} \DecValTok{1} \SpecialCharTok{|}\NormalTok{ dice.rolls }\SpecialCharTok{\textgreater{}} \DecValTok{4}\NormalTok{)}
\end{Highlighting}
\end{Shaded}

\begin{verbatim}
## [1] 9
\end{verbatim}

Additionally, since logical operators work with text strings, it is
possible to use them with \texttt{sum()} to return the number of heads
if heads is represented by, for example, \texttt{H}.

\begin{Shaded}
\begin{Highlighting}[]
\CommentTok{\#set the seed for a pseudo{-}random sample}
\FunctionTok{set.seed}\NormalTok{(}\DecValTok{2018}\NormalTok{)}
\NormalTok{outcomes }\OtherTok{=} \FunctionTok{sample}\NormalTok{(}\FunctionTok{c}\NormalTok{(}\StringTok{"T"}\NormalTok{, }\StringTok{"H"}\NormalTok{), }\AttributeTok{size =} \DecValTok{10}\NormalTok{, }\AttributeTok{prob =} \FunctionTok{c}\NormalTok{(}\FloatTok{0.4}\NormalTok{, }\FloatTok{0.6}\NormalTok{), }\AttributeTok{replace =} \ConstantTok{TRUE}\NormalTok{)}
\NormalTok{outcomes}
\end{Highlighting}
\end{Shaded}

\begin{verbatim}
##  [1] "H" "H" "H" "H" "H" "H" "T" "H" "T" "H"
\end{verbatim}

\begin{Shaded}
\begin{Highlighting}[]
\FunctionTok{sum}\NormalTok{(outcomes }\SpecialCharTok{==} \StringTok{"H"}\NormalTok{)  }\CommentTok{\#number of heads}
\end{Highlighting}
\end{Shaded}

\begin{verbatim}
## [1] 8
\end{verbatim}

\hypertarget{for-loops}{%
\subsubsection{\texorpdfstring{\texttt{for}
Loops}{for Loops}}\label{for-loops}}

A loop allows for a set of code to be repeated under a specific set of
conditions; the \texttt{for} loop is one of the several types of loops
available in R.

A \texttt{for} loop has the basic structure
`\texttt{for(\ counter\ )\ \textbackslash{}\{\ instructions\ \textbackslash{}\}}.
The loop below will calculate the squares of the integers from 1 through
5.

\begin{itemize}
\item
  Prior to running the loop, an empty vector \texttt{squares} is created
  to store the results of the loop. This step is referred to as
  initialization.
\item
  The counter consists of the index variable that keeps track of each
  iteration of the loop; the index is typically a letter like
  \texttt{i}, \texttt{j}, or \texttt{k}, but can be any sequence such as
  \texttt{ii}. The index variable is conceptually similar to the index
  of summation \(k\) in sigma notation (\(\sum_{k = 1}^n\)). In the
  example below, the counter can be read as ``for every \(k\) in 1
  through 5, repeat the following instructions\ldots{}''
\item
  The instructions are enclosed within the pair of curly braces. For
  each iteration, the value \(k^2\) is to be stored in the \(k^{th}\)
  element of the vector \texttt{squares}. The empty vector
  \texttt{squares} was created prior to running the loop using
  \texttt{vector()}.
\item
  So, for the first iteration, R sets \(k = 1\), calculates \(1^2\), and
  fills in the first element of \texttt{squares} with the result. In the
  next iteration, \(k = 2\)\ldots{} and this process repeats until
  \(k = 5\) and \(5^2\) is stored as the fifth element of
  \texttt{squares}.
\item
  After the loop is done, the vector \texttt{squares} is no longer empty
  and instead consists of the squares of the integers from 1 through 5:
  1, 4, 9, 16, and 25.
\end{itemize}

\begin{Shaded}
\begin{Highlighting}[]
\CommentTok{\#create empty vector to store results (initialize)}
\NormalTok{squares }\OtherTok{=} \FunctionTok{vector}\NormalTok{(}\StringTok{"numeric"}\NormalTok{, }\DecValTok{5}\NormalTok{)}
\CommentTok{\#run the loop}
\ControlFlowTok{for}\NormalTok{(k }\ControlFlowTok{in} \DecValTok{1}\SpecialCharTok{:}\DecValTok{5}\NormalTok{)\{}
  
\NormalTok{  squares[k] }\OtherTok{=}\NormalTok{ k}\SpecialCharTok{\^{}}\DecValTok{2}
  
\NormalTok{\}}
\CommentTok{\#print the results}
\NormalTok{squares}
\end{Highlighting}
\end{Shaded}

\begin{verbatim}
## [1]  1  4  9 16 25
\end{verbatim}

Of course, the same result could be easily achieved without a
\texttt{for} loop:

\begin{Shaded}
\begin{Highlighting}[]
\NormalTok{(}\DecValTok{1}\SpecialCharTok{:}\DecValTok{5}\NormalTok{)}\SpecialCharTok{\^{}}\DecValTok{2}
\end{Highlighting}
\end{Shaded}

\begin{verbatim}
## [1]  1  4  9 16 25
\end{verbatim}

The \texttt{for} loop is a useful tool when the set of instructions to
be repeated is more complicated, such as in the coin tossing scenario
from the lab. The loop from Question 2 is reproduced here for reference.

\begin{itemize}
\item
  The loop is set to run for every \(k\) from 1 to the value of
  \texttt{number.replicate}, which has been previously defined as 50.
\item
  There are two vectors defined within the curly braces.

  \begin{itemize}
  \item
    The first vector, \texttt{outcomes.replicate}, consists of the
    outcomes for the tosses in a single replicate (i.e., iteration of
    the loop). The number of tosses in a single experiment has been
    defined as 5. These values are produced from \texttt{sample()}. This
    vector's values are re-populated each time the loop is run.
  \item
    The second vector, \texttt{outcomes}, is created by calculating the
    sum of \texttt{outcomes.replicate} for each iteration of the loop.
    Once the loop has run, \texttt{outcomes} is a record of the number
    of \texttt{1}'s that occurred for each iteration.
  \end{itemize}
\item
  In the language of the coin tossing scenario: the loop starts by
  tossing 5 coins, counting how many heads occur, then recording that
  number as the first element of \texttt{outcomes}. Next, the loop
  tosses 5 coins again and records the number of heads in the second
  element of \texttt{outcomes}. The loop stops once it has repeated the
  experiment (of tossing 5 coins) 50 times.

  \begin{itemize}
  \item
    The distinction between the \texttt{outcomes.replicate} vector and
    the \texttt{outcomes} vector, for each run of the loop, is that the
    \texttt{outcomes.replicate} vector stores the exact sequence of
    results while \texttt{outcomes} only records the number of heads.
  \item
    For this particular problem, the only information needed is the
    number of heads; thus, one can think of \texttt{outcomes} as
    summarizing the relevant information from the experiment.
  \end{itemize}
\end{itemize}

\begin{Shaded}
\begin{Highlighting}[]
\ControlFlowTok{for}\NormalTok{(k }\ControlFlowTok{in} \DecValTok{1}\SpecialCharTok{:}\NormalTok{number.replicates)\{}
  
\NormalTok{  outcomes.replicate }\OtherTok{=} \FunctionTok{sample}\NormalTok{(}\FunctionTok{c}\NormalTok{(}\DecValTok{0}\NormalTok{, }\DecValTok{1}\NormalTok{), }\AttributeTok{size =}\NormalTok{ number.tosses,}
                              \AttributeTok{prob =} \FunctionTok{c}\NormalTok{(}\DecValTok{1} \SpecialCharTok{{-}}\NormalTok{ prob.heads, prob.heads), }\AttributeTok{replace =} \ConstantTok{TRUE}\NormalTok{)}
  
\NormalTok{  outcomes[k] }\OtherTok{=} \FunctionTok{sum}\NormalTok{(outcomes.replicate)}
  
\NormalTok{\}}
\end{Highlighting}
\end{Shaded}


\end{document}
